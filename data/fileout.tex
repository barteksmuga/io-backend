\PassOptionsToPackage{unicode=true}{hyperref} % options for packages loaded elsewhere
\PassOptionsToPackage{hyphens}{url}
%
\documentclass[]{article}
\usepackage{lmodern}
\usepackage{amssymb,amsmath}
\usepackage{ifxetex,ifluatex}
\usepackage{fixltx2e} % provides \textsubscript
\ifnum 0\ifxetex 1\fi\ifluatex 1\fi=0 % if pdftex
  \usepackage[T1]{fontenc}
  \usepackage[utf8]{inputenc}
  \usepackage{textcomp} % provides euro and other symbols
\else % if luatex or xelatex
  \usepackage{unicode-math}
  \defaultfontfeatures{Ligatures=TeX,Scale=MatchLowercase}
\fi
% use upquote if available, for straight quotes in verbatim environments
\IfFileExists{upquote.sty}{\usepackage{upquote}}{}
% use microtype if available
\IfFileExists{microtype.sty}{%
\usepackage[]{microtype}
\UseMicrotypeSet[protrusion]{basicmath} % disable protrusion for tt fonts
}{}
\IfFileExists{parskip.sty}{%
\usepackage{parskip}
}{% else
\setlength{\parindent}{0pt}
\setlength{\parskip}{6pt plus 2pt minus 1pt}
}
\usepackage{hyperref}
\hypersetup{
            pdfborder={0 0 0},
            breaklinks=true}
\urlstyle{same}  % don't use monospace font for urls
\setlength{\emergencystretch}{3em}  % prevent overfull lines
\providecommand{\tightlist}{%
  \setlength{\itemsep}{0pt}\setlength{\parskip}{0pt}}
\setcounter{secnumdepth}{0}
% Redefines (sub)paragraphs to behave more like sections
\ifx\paragraph\undefined\else
\let\oldparagraph\paragraph
\renewcommand{\paragraph}[1]{\oldparagraph{#1}\mbox{}}
\fi
\ifx\subparagraph\undefined\else
\let\oldsubparagraph\subparagraph
\renewcommand{\subparagraph}[1]{\oldsubparagraph{#1}\mbox{}}
\fi

% set default figure placement to htbp
\makeatletter
\def\fps@figure{htbp}
\makeatother


\date{}

\begin{document}

Termodynamika jest gałęzią nauki zajmującą się, ogólnie biorąc,
przekształceniami energii z jednej postaci w drugą oraz właściwościami
ciał biorących udział w tych procesach. W szczególności termodynamika
techniczna zajmuje się zależnościami między ciepłem i pracą oraz
właściwościami fizykochemicznymi systemów materialnych.

Ciepło właściwe jest to własność termofizyczna płynu. Średnie ciepło
właściwe:

\texttt{(przemiany,~T,P)~(1..31)}

zależne jest od rodzaju przemiany, temperatury i ciśnienia, przy czym
zależność od temperatury jest silniejsza niż od ciśnienia. Ta ogólna
zależność może być ograniczona do uzależnienia ciepła właściwego tylko
od przemiany, jak to ma miejsce w gazach doskonałych, lub od przemiany i
temperatury, co charakteryzuje gazy półdoskonałe. Szczególną rolę w
termodynamice odgrywają ciepła właściwe dwu przemian: • odbywającej się
przy stałej objętości, czyli izochorycznej: Cv - ciepło właściwe przy
stałej objętości, J/ kg·K

• odbywającej się przy stałym ciśnieniu, czyli izobarycznej: Cp - ciepło
właściwe przy stałym ciśnieniu, J/kg·K

Stała objętość: powietrze ogrzewane w plastikowym/szklanym naczyniu.
Stałe ciśnienie: ogrzewany/ochładzany balonik. Działanie otoczenia na
układ termodynamiczny nazywane jest pracą, jeżeli wynik tego działania
można sprowadzić tylko do zmiany położenia ciężaru, znajdującego się
poza układem, względem poziomu odniesienia. Działania otoczenia na układ
zamknięty, które nie mogą być zaliczane do różnego rodzaju prac, są
nazywane w termodynamice zewnętrznym ciepłem układu, a sposób w jaki
jest przekazywane to ciepło wymianą ciepła, przepływem ciepła lub
przenoszeniem ciepła. Wymiana ciepła jest realizowana na różne pod
względem fizycznym sposoby: przewodzenie, konwekcję i radiacyjną wymianę
ciepła. W układach otwartych energia jest przekazywana również przez
granice układu wraz z przepływem substancji w postaci energii
kinetycznej, potencjalnej lub entalpii, zwanej często w technice energią
cieplną. W przypadku gdy zjawiska wymiany ciepła zmieniają się w miarę
upływu czasu, występuje nieustalona wymiana ciepła, a gdy nie zmieniają
się w czasie -- ustalona wymiana ciepła. Wymiana ciepła występuje pod
wpływem różnicy temperatury. Przy rozpatrywaniu zjawisk wymiany ciepła
jest konieczna znajomość pola temperatury. Polem temperatury jest
nazywany zbiór wartości temperatury we wszystkich punktach
rozpatrywanego ciała w danej chwili. Pole temperatury T jest określone
przez zależność temperatury od współrzędnych przestrzeni (np.
kartezjańskich x, y, z) oraz od czasu t: T=f(x,y,z,t) ( 1.1) Gdy pole
temperatury zmienia się w czasie, wymiana ciepła jest nieustalona.
Wymiana ciepła jest ustalona, gdy pole temperatury nie zmienia się w
czasie, czyli gdy temperatura jest tylioi:o funkcją współrzędnych
przestrzeni T = F( x,y,z), = 0 (1.2) W zależności od liczby
współrzędnych przestrzeni, w kierunku których zmienia się temperatura,
pole temperatury i wymiana ciepła są jedno-, dwu- lub trójwymiarowe.
Zbiór punktów przestrzeni o jednakowej temperaturze tworzy powierzchnię
izotermiczną. Zbiór punktów powierzchni o jednakowej temperaturze tworzy
linię izotermiczną. Ponieważ w jednym punkcie przestrzeni nie mogą
występować różne temperatury tej samej substancji, powierzchnie (lub
linie) izotermiczne nie przecinają się, lecz tworzą powierzchnie (lub
linie) zamknięte albo kończą się na powierzchni ciała. Zarówno w
termodynamice, jak i w wymianie ciepła ciepło Q {[}kJ{]} jest wielkością
skalarną, chociaż mówimy o kierunku przepływu ciepła od wyższej do
niższej temperatury. Stosunek elementarnej ilości ciepła dQ do czasu
trwania wymiany tej ilości ciepła jest nazywany strumieniem ciepła
(wyrażanym w W)

a w warunkach ustalonych:

Strumień ciepła, podobnie jak i ciepło, jest wielkością skalarną. Po
odniesieniu strumienia ciepła do jednostki pola powierzchni A (ściśle
zorientowanej w przestrzeni) otrzymuje się wektor zwany gęstością
strumienia ciepła (wyrażoną w ). Jest to wektor prostopadły do
powierzchni izotermicznej, skierowany zgodnie ze spadkiem temperatury, o
module równym stosunkowi elementarnego strumienia ciepła dQ do
elementarnego pola powierzchni dA, przez którą strumień ten przepływa

lub w zapisie wektorowym:

W szczególnym przypadku gęstość strumienia ciepła w każdym punkcie
rozpatrywanej powierzchni jest taka sama i wynosi:

Punktem wyjścia do określenia pola temperatury w ciele stałym jest
równanie różniczkowe przewodzenia ciepła, które otrzymuje się na
podstawie równania bilansu energii dla elementarnej objętości
substancji. W prostokątnym układzie współrzędnych x, y, z elementarna
objętość substancji jest przedstawiona jako prostopadłościan o bokach
dx, dy, dz i objętości dV = dxdydz (rys. 1.5).

Y Przy stałym ciśnieniu bilans energii dla ciała stałego przechodzi w
bilans entalpii. Przyrost entalpii substancji (o gęstości p i cieple
właściwym cP) zawartej w elementarnej objętości d V zmienia się w czasie
dt o wartość

Ta zmiana entalpii wywołana jest doprowadzaniem ciepła z zewnątrz przez
przewodzenie i ewentualnie doprowadzaniem ciepła od wewnątrz, gdy
istnieją wewnętrzne źródła ciepła. Strumień ciepła przewodzony wzdłuż
osi x zmienia się na odległości dx o

Analogicznie otrzymuje się zmiany strumieni ciepła w pozostałych
kierunkach osi współrzędnych. Pozwala to na zestawienie równania bilansu
entalpii dla izotropowego ciała stałego w postaci równania

Wyprowadzone niżej wzory na jednostkową energię cieplną u i entalpię h =
u + Pv nazywane są też kalorycznymi równaniami stanu. Entalpia
reprezentuje energię cieplną w równaniach bilansowych systemów
otwartych, z jakimi bardzo często mamy do czynienia w technice. W
ogólnym przypadku substancji prostej, kiedy stan tej substancji jest w
zupełności określony przez 2 parametry: T, P lub T, v (ew. P, v),
energia cieplna i entalpia są funkcjami dwu zmiennych: u = f(T,v) h =
f(T,P) Gazy doskonale i półdoskonale Na energię cieplną gazów
doskonałych i półdoskonałych składa się wyłącznie energia kinetyczna
ruchów molekuł i atomów, pomija się energię potencjalną oddziaływania
tych cząstek na siebie. Chodzi tu o energie ruchu postępowego i
obrotowego a w gazach półdoskonałych jeszcze o energię ruchu drgającego
atomów w molekule. Energia cieplna 1 kg gazu doskonałego lub
półdoskonałego może być najprościej wyznaczona z równania I zasady
termodynamiki:

zastosowanego do przemiany izochorycznej, w której dla v = const jest dv
= 0 tak że równanie I Z.T. upraszcza się do postaci

Wynika z tego, że ciepło doprowadzone do ciała zachowującego stałą
objętość w całości podwyższa energię wewnętrzną tego ciała. Ale ciepło
przemiany izochorycznej określa wzór dqv = cvdT (3.36) Przyrównując
stronami (3.35) i (3.36), otrzymuje się:

w którym indeks v oznacza różniczkowanie przy v = const. Dla gazów
doskonałych jest oczywiście cv = const f(T) i całkowanie równania (3.37)
w granicach od stanu odniesienia (normalnego, standardowego) do stanu
bieżącego daje

czyli u - u 0 = cv· (T-T0 ) = cv· ( t - t 0 ) (3.39) Przyjmuje się
temperaturę stanu odniesienia T0 = 273,15 K= 0°C i wtedy:

\end{document}
