\PassOptionsToPackage{unicode=true}{hyperref} % options for packages loaded elsewhere
\PassOptionsToPackage{hyphens}{url}
%
\documentclass[]{article}
\usepackage{lmodern}
\usepackage{amssymb,amsmath}
\usepackage{ifxetex,ifluatex}
\usepackage{fixltx2e} % provides \textsubscript
\ifnum 0\ifxetex 1\fi\ifluatex 1\fi=0 % if pdftex
  \usepackage[T1]{fontenc}
  \usepackage[utf8]{inputenc}
  \usepackage{textcomp} % provides euro and other symbols
\else % if luatex or xelatex
  \usepackage{unicode-math}
  \defaultfontfeatures{Ligatures=TeX,Scale=MatchLowercase}
\fi
% use upquote if available, for straight quotes in verbatim environments
\IfFileExists{upquote.sty}{\usepackage{upquote}}{}
% use microtype if available
\IfFileExists{microtype.sty}{%
\usepackage[]{microtype}
\UseMicrotypeSet[protrusion]{basicmath} % disable protrusion for tt fonts
}{}
\IfFileExists{parskip.sty}{%
\usepackage{parskip}
}{% else
\setlength{\parindent}{0pt}
\setlength{\parskip}{6pt plus 2pt minus 1pt}
}
\usepackage{hyperref}
\hypersetup{
            pdfborder={0 0 0},
            breaklinks=true}
\urlstyle{same}  % don't use monospace font for urls
\setlength{\emergencystretch}{3em}  % prevent overfull lines
\providecommand{\tightlist}{%
  \setlength{\itemsep}{0pt}\setlength{\parskip}{0pt}}
\setcounter{secnumdepth}{0}
% Redefines (sub)paragraphs to behave more like sections
\ifx\paragraph\undefined\else
\let\oldparagraph\paragraph
\renewcommand{\paragraph}[1]{\oldparagraph{#1}\mbox{}}
\fi
\ifx\subparagraph\undefined\else
\let\oldsubparagraph\subparagraph
\renewcommand{\subparagraph}[1]{\oldsubparagraph{#1}\mbox{}}
\fi

% set default figure placement to htbp
\makeatletter
\def\fps@figure{htbp}
\makeatother


\date{}

\begin{document}

\begin{itemize}
\tightlist
\item
  Start each line
\item
  with an \href{Wikipedia:asterisk}{asterisk} (*).

  \begin{itemize}
  \tightlist
  \item
    More asterisks give deeper

    \begin{itemize}
    \tightlist
    \item
      and deeper levels.
    \end{itemize}
  \end{itemize}
\item
  Line breaks\\
  don't break levels.

  \begin{itemize}
  \item
    \begin{itemize}
    \tightlist
    \item
      But jumping levels creates empty space.
    \end{itemize}
  \end{itemize}
\end{itemize}

Any other start ends the list.

\begin{itemize}
\tightlist
\item
  combine bullet list

  \begin{itemize}
  \tightlist
  \item
    with definition
  \end{itemize}
\end{itemize}

\begin{description}
\item[]
\begin{description}
\tightlist
\item[]
- definition
\end{description}
\end{description}

\begin{itemize}
\item
  \begin{itemize}
  \tightlist
  \item
    creates empty space
  \end{itemize}
\end{itemize}

\begin{itemize}
\tightlist
\item
  combine bullet list

  \begin{itemize}
  \tightlist
  \item
    with definition
  \end{itemize}

  \begin{description}
  \tightlist
  \item[]
  - definition
  \end{description}

  \begin{itemize}
  \tightlist
  \item
    without empty spaces
  \end{itemize}
\end{itemize}

\begin{itemize}
\tightlist
\item
  bullet list
\end{itemize}

\begin{description}
\tightlist
\item[]
- definition

\begin{itemize}
\tightlist
\item
  sublist that doesn't create empty
\item
  spaces after definition
\end{itemize}
\end{description}

\end{document}
